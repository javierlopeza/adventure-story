\documentclass{article}
\usepackage[utf8]{inputenc}

\begin{document}
\Huge Adventure Story Project

\huge So, how did we all get here?

\normalsize

\subsection*{Chapter 1: \textit{May}}

Hi, I'm May from Malaysia. I was born and raised on an island called Penang, which is located on the northwest coast of Peninsular  Malaysia. As an only child, I naturally became closer to my parents than my peers would as my parents undertook the roles of siblings - they would play Scrabble with me, listen to me rant whenever I have any problems as well as provide me with emotional support. 

One interesting fact about me is that I speak five different languages. My parents, who are third-generation Malaysian Chinese, have been speaking in Hokkien dialect, a dialect originating from Southwestern China,  with me ever since I’m young. I went to a kindergarten which taught me three different languages, i.e. English, Mandarin and Malay, Malaysia’s national language. My nanny, who babysitted me from birth till I was 10 years old, is a first generation Malaysian-Chinese immigrant from Guangzhou, China.  I picked up Cantonese from her as she would speak Cantonese with her children and watch Hong Kong dramas extensively. 

I went to an all-girls school from elementary through middle school. I personally feel that going to an all-girls school during my developmental years has helped me build confidence and succeed academically. I held leadership positions in school, represented my school in Math Olympiad and won school award for scoring highest score in Mathematics. Gender stereotyping was pretty much unheard of in my school- girls were never compared to boys in terms of achievements in Math and Science. My schoolmates and I never felt that we were incapable of doing something or doing well in traditionally male-dominated subjects, just because we are females. 

Growing up in a low-middle income class family, I barely had the chance to travel. As a child, I loved watching documentaries on National Geographic channel as they gave me a glimpse into lives of people living across the globe. I longed to experience a different culture, to play with snow, to interact with people of different nationalities and to explore new places. I realized from a young age that the only way for me to achieve my dreams is to obtain a scholarship which funds my education abroad. I worked really hard during middle school to ace the Malaysian public examination as my scores would determine if I get a scholarship. 

My dream turned into reality when I emerged as the Top 50 students out of 480,000 students in Malaysia who took the examination. I secured a full-scholarship, provided by the Malaysian government, which funds my education abroad. I had the option to choose the university I wish to attend, as long as they are ranked among the Top 50 universities in the world , according to the TIMES higher education rankings. I successfully gain admissions to top schools in the UK and UC Berkeley. I was debating between attending London School of Economics and UC Berkeley. In the end, I decided to come to the States as I wanted a more holistic education, which the liberal arts system provides. I am extremely grateful for the opportunity to attend UC Berkeley. The past four years I had in Berkeley has been extremely enriching and eye-opening. I am glad to call Berkeley my home for four years. 


\subsection*{Chapter 2: \textit{Hoaian}}

Being born in San Jose, the heart of the Silicon Valley, I grew up surrounded by towering tech company headquarters. With technological centers of Apple or Google just a short car ride away, I knew full well  the immense power and reach of computer science. I saw it on every long car. I felt it with awe. In both senses of the word, with all of the wonder, and, especially, all of the fear. Growing up as a low-income student, with parents who neither finished college nor spoke English fluently, I convinced myself that Computer Science was an impossibility. It took me a full 17 years until I got over this misconception and found my first CS mentor and my headway into the guarded walls of Computer Science. It was my senior year of high school, and I became friends with Michael, a junior who was just a Computer Science Genius. Together, we started the Oak Grove Robotics Club. I’ve been obsessed with coding ever since. Coming here to Berkeley and taking CS61A, however, was a completely new experience, and I would say, my first real taste of Computer Science. Though soul-crushing and tear-filled, it was a euphoric experience. I want to pursue that feeling, and apply my newly learned knowledge into something real and tangible, with an actual real world impact. That’s why I want to get into product development in general. I want to join CodeBase specifically because of your mentorship program. I learned from my experiences with Michael and the Robotics Club that Computer Science isn’t something to be learned without guidance. I’m still just now getting over my fear of Stack Overflow and its ever menacing, capital ‘L’ Lingo. I'm excited to learn everything I can from being here at Berkeley.

\subsection*{Chapter 3: \textit{Javier}}

How did I get here? Actually it was a long journey. I come from Chile, from the west coast of South America, and for someone like me, leaving the country and studying abroad is really an adventure. I left my family behind, thousands of miles away, and set out on a journey in search of knowledge and new opportunities. This adventure has allowed a character like me to discover new cultures and exceptional people that will mark life forever.

It all started a year ago, when I was presented with the opportunity of my life: to go to study at the University of California, Berkeley. It was an opportunity that I decided to take without thinking twice, because I was deciding to go to one of the best universities in the world.

Months after my application, I received the expected letter of acceptance and after that I had to prepare all the details of the trip. The farewell is difficult. Friends, family and many people that I will not see in a long time. I hope to come back and join them to share all my experiences and adventures.

This adventure is just beginning, but for now there have been many experiences that I have faced and will never forget. I have had the good fortune to cross the Golden Gate Bridge by bicycle, while water sports enthusiasts perform kitesurfing and cross the bridge below spectacularly. I have been able to walk the inclined streets of San Francisco and appreciate the multitude of cultures that share in every corner of the city.

I have had to face hunger, cold and fatigue, but there is still a long way to go in this adventure. Great feats and vicissitudes are coming to avoid succumbing to the remoteness of the home. But I am convinced that in the future I will return home and I will be able to tell with great happiness all the stories of my adventures in California.

That's how I got here.


\subsection*{Chapter 4: \textit{Christie}}

“Shoot!” I curse to myself as I run to turn off the stove. I had accidentally left my brussel sprouts too long on the pan. I return to my kitchen counter, where countless jars of herbs and spices riddle the space, neighboring bowls filled with hot, smoky rub, and sauces that are waiting to be marinated with. Smells muddling into the crisp air; smells of sizzling fresh garlic, searing caramelized brussel sprouts. Pots and pans continue to fill the sink, each once holding its own little creation: five-spice marinated chicken wings, lemongrass and cilantro black rice. \par
This apparent discord is not just spilled bowls of rustic vegetable soup or two too many clouds of flour. It is not cut fingertips and burned palms or the possibility that I might have burned the house down. It is, rather, some sort of harmonious chaos of food. It is the explosion of flavors that the nutty pesto sauce gives to the cucumber and the kick of heat the cayenne gives the rub. It is the sweet aroma of lavender infused sugar and apple pie fresh out of the oven. It is my kitchen. 
It is not like this everyday; most days run smoothly with ingredients coming in and food coming out. But there are still times in which things get out of hand, times in which I fail, and times in which I admittedly experiment too much. \par
My time in the kitchen has taught me the fragility of cooking, how one more teaspoon of vanilla could make such a huge difference or how fifteen extra seconds on the pan could overcook scallops. It has taught me creativity, that desserts can be savory and that yes, oranges can actually go with chocolate. But more importantly, it has inspired my need for experimentation, my need for something more appetizing, for something more enticing, for something new and exciting. \par
My world inspired me to step out of my small box, in which I had once lived my whole life before, during high school, leading into college. My life became my kitchen. I was able to taste new experiences and devour adventures. These experiences only intensify my thirst something newer and greater that my world inspired. I want to go out and try new things and gain more experience. I want to make mistakes and maybe set of alarms and maybe burn my brussel sprouts from time to time, and I definitely have done both here. I might’ve burnt my house down, but, I think I found something great here at Berkeley I’ll carry through the rest of my life.


\end{document}
